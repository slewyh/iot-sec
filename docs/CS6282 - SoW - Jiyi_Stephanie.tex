\documentclass[a4paper, 11pt]{article}
\usepackage{comment} % enables the use of multi-line comments (\ifx \fi) 
\usepackage{lipsum} %This package just generates Lorem Ipsum filler text. 
\usepackage{fullpage} % changes the margin
\usepackage{filecontents}
\begin{filecontents*}{testing.bib}
@ARTICLE{893287, 
author={}, 
journal={IEEE Std. 1516-2000}, 
title={{IEEE Standard for Modeling and Simulation {(M\&S)} High Level Architecture {(HLA)} - Framework and Rules}}, 
year={2000}, 
volume={}, 
pages={i -22}, 
doi={10.1109/IEEESTD.2000.92296}, 
}
\end{filecontents*}

\begin{document}
%Header-Make sure you update this information!!!!
\noindent
\large\textbf{CS6282: Statement of Work} \hfill  \\
\normalsize Spring 2019 \hfill Teammates: Stephanie Lew, Zhang JiYi \\
Prof. Jun Han \hfill  Date: XX/02/19 \\


\section*{Introduction/Background}
Put your Problem statement here! Example of a Citation\cite{7867716}. See \cite{893287} for more info.
\section*{Objectives}
\lipsum[2]

\section*{Scope of Work}
\lipsum[3]

\section*{Proposed Milestones}
\lipsum[4]
% to comment sections out, use the command \ifx and \fi. Use this technique when writing your pre lab. For example, to comment something out I would do:
%  \ifx
%	\begin{itemize}
%		\item item1
%		\item item2
%	\end{itemize}	
%  \fi

\section*{Attachments}
%Make sure to change these
Lab Notes, HelloWorld.ic, FooBar.ic
%\fi %comment me out

\bibliographystyle{IEEEtran}
\bibliography{IEEEabrv,testing}
\end{document}
















