\documentclass[a4paper, 11pt]{article}
\usepackage{comment} % enables the use of multi-line comments (\ifx \fi) 
\usepackage{lipsum} %This package just generates Lorem Ipsum filler text. 
\usepackage{fullpage} % changes the margin
\usepackage{float}

\begin{document}
\noindent
\large\textbf{CS6282: Statement of Work} \hfill  \\
\normalsize Spring 2019 \hfill Teammates: Stephanie Lew, Zhang Jiyi \\
Prof. Jun Han \hfill  Date: 08/03/19 \\

{\bf Keywords:} Continuous Authentication, Multi-modal biometrics
\section*{Executive Summary}
It is well known that single sign-in solutions involving passwords and tokens are not very secure. Behavioral biometrics promises to be stronger authenticators as user behaviors can be learned from biometric features collected from embedded sensors in mobile devices. However, these methods still does not solve the single point-of-entry problem and is vulnerable to attacks. On the other hand, continuous authentication ensures that users’ identity are verified continuously and strengthens the ability of a system to lock out imposters, even when the recognition ability of a single sample is not high. The use of multi-modal biometrics such as fusing face, voice, teeth modalities, or touchscreen gestures and textual information has proven to be more effective overall authenticators as compared to a single biometric authenticator alone. Our work  tries to solve the problem of seamlessly authenticating users across multiple devices in a smart home setting by continuous authentication. We realise that the existing methods may not work in a complex setting like a smart home, where not all mobile devices are equipped with the same sensing modality. Furthermore, processing time costs increase as with the increase in the number of biometric modalities considered for continuous authentication. Thus, we propose a multi-modal continuous authentication system that is capable of fusing time-stamped information (e.g. face, voice and location) collected from devices of different common biometric sensing modalities, that is not limited by device hardware configurations. We envision that our research can open up new works for evaluating which combination of sensing modalities can be most effective for smart home use. 

\section*{Introduction}
Mobile devices are becoming an inherent part of our daily lives. Several applications in smartphones for example, provide us with the means to communicate, plan work and organize our lives. The smartphones collect and compile a growing amount of private information, to which access needs to be controlled to protect the privacy of the users. \\\\
\noindent
To prevent unauthorized access to private data in smartphones, smartphone manufacturers employ traditional access control mechanisms using passcode, lock pattern and fingerprint. Previous work~\cite{} has shown that these methods are limited in terms of the security, usability and cost. In response, researchers propose continuous and passive authentication as an enhancement to existing schemes~\cite{}. A continuous and passive authentication system continually senses a user’s interactions with the device and authenticates the user in runtime based on behavioral data, such as touchscreen gestures, walking gait, face and voice modalities~\cite{}. These characteristics promise to be stronger than passwords alone.\\\\
\noindent
Such information can be collected from different sources of context, like a device, a network, online resources or the environment, which can then be factored into an authentication decision, either as separate authentication factors or as an authentication adaptation, augmenting the decision to grant a user access to a given resource. While considering a high number of attributes for authentication would increase security and reliability, processing time cost would also increase. \cite{}suggests that any proposed solution should contain contextual attributes that do not require direct user interaction and the verification functions should work in the background without any response delay. \\\\
\noindent
In this project, we propose a multi-modal biometric approach toward authenticating users continuously and transparently in smart home environments.

\section*{Problem Statement}
Our aim is to build a component-based system which combines the sensor capability and computation power of multiple different smart home devices to provide secure, robust, reliable and convenient authentication. In this system, every smart home device runs a submodule which takes in the device’s sensor input and gives a score describing the likelihood of owner presence. All the submodules connect to the core processing component through an interface. This core component constantly takes in scores from different devices and their respective timestamps to conduct real-time evaluation and produce the authentication decision. The evaluation should be done in such a way that the score itself is independent of the sampling rate and type of sensor activated at the moment of authentication. A higher sampling frequency increases the confidence of the decision while a lower sampling frequency does the opposite. Adaptive sampling is applied to maintain the confidence level above a minimum threshold while avoiding hogging the computation resources for prolonged time period.\\\\
\noindent
Many smart home devices like security cameras and voice assistants can be controlled through end-user devices like smartphones and tablets. Notably, current passcode-based and cryptographic-based authentication methods for such devices are neither convenient nor secure. Though biometric authentication methods such as fingerprint verification and facial verification have become more popular recently, they are often used as a simple replacement of the password authentication, which is deemed unsafe. Hence, the issue of single point of failure still remains unresolved. When a mobile device using such kind of authentication is in possession of an adversarial, the attacker will have enough time and resource to achieve partial or total break eventually, and may result in serious consequences.

\section*{Related Work}
A number of researches have upheld the need for more innovative authentication methods that aim to balance the trade-off between security and convenience. These authentication approaches commonly fall into three categories: knowledge-based, object-based and physiological.\\\\
\noindent
One type of knowledge-based authentication is Single sign-on, which relies on the user to provide one login credential to access many other applications~\cite{Abdulwahid2016} . This increases convenience and reduce the burden of the user having to remember many passwords. They assume that the user’s identity is legitimate throughout the access session, which is not necessarily the case (an imposter may try to use the applications after the initial login). \\\\
\noindent
An object-based approach can be taken to overcome the drawbacks of knowledge-based approach like passwords. For example, Token-based authentication in a form of hardware or software tokens are used for logging in to the service. A software token such as the Google Authenticator~\cite{}, installed as a software on a device, issues a new password that changes with every access time. This method removes the burden of remembering or choosing passwords.\\\\
\noindent
Some researchers realized the importance of continuity and context in security implementations~\cite{Nakanishi:2016:UBC:3015166.3015180, Sitara:2016:DVT:3009206.3009228,https://calhoun.nps.edu/handle/10945/1011,Janakiraman:2005:UCF:1042449.1043974,ebb6df849f174c33bc9df41b799a296d}. They proposed the idea of behavioral-based authentication. Nonetheless, the ‘side channel’ nature of behavioral based method made it inadequate to act as a strong authenticator alone. Therefore, it becomes crucial to implement a mechanism which combines weak authenticators together to provide strong authentication.

\section*{Project Plan}
\begin{enumerate}
\item We will start from making submodules for a specific set of sensors first. These submodules will be the ‘weak authenticators’ which get combined to provide ‘strong’ authentication in the later stages. In particular, we will make face verification module, speaker recognition module and GPS location module. This approach will give a good coverage for sensors on most mobile and IoT devices, as cameras, microphones and GPS are the most common sensors available.  
\item Each ‘weak authenticator’ will provide a score and a timestamp of this score whenever a ‘authentication’ command is issued to it. We will design unified APIs for submodules so each module can just ‘plug-and-play’. Addition of new modules in the future will be easy too.
\item When we have the three submodules we need for prototyping, we will design the core component. The core component will be the brain of the whole system. It will issue authentication command to each submodule when necessary, and it will give an authentication decision based on the scores it collected.
\item The design of the core component will also start from simple approach such as sliding window method so that we can make things work first. We can then move on to use machine learning methods such as LSTM. Eventually, we hope to come up with our own solution for the irregular-spaced time series problem which is crucial in this context of continuous authentication.
\item Finally, we would like to think about how users can interact with this system. If we have time, we can come up with a flow from recording biometric data to setting policy and eventually activating the authentication system.
\end{enumerate}

\section*{Project Timeline}
\begin{table}[H]
\begin{tabular}{|l|l|}
\hline
Week             & Task                                                                         \\ \hline
Feb. 17 - 23     & Literature Survey on Continuous Authentication                               \\ \hline
Feb. 24 - Mar. 2 & Literature Survey on state-of-art facial verification and speaker recognizer \\ \hline
Mar. 3 - 9       & Develop submodules for individual authenticator                              \\ \hline
Mar. 10 - 16     & Study irregular-spaced time series                                           \\ \hline
Mar. 17 - 30     & Develop the core component                                                   \\ \hline
Mar. 31 - Apr. 6 & Link everything together using unified APIs                                  \\ \hline
Apr. 7 - 13      & Design user interface                                                        \\ \hline
Apr. 14 - 21     & Prepare final report                                                         \\ \hline
\end{tabular}
\end{table}

\section*{Team Responsibilities}

\section*{Potential Risks}
\begin{enumerate}
\item The difficult part of our project is the design of the core component which combines scores from different authenticators meaningfully. It will be hard to give a rigorous mathematical formulation. We need to have a rigorous estimation of error bound based on the confidence threshold, sampling distribution and individual authenticator’s precision and recall. Otherwise, our contribution will be less significant.
\item We will have good amount of data for us to play around with and build the individual authenticators. However, it will be hard to collect data from actual smart home setting and test the whole system.
\end{enumerate}


\section*{Backup Plan}

% to comment sections out, use the command \ifx and \fi. Use this technique when writing your pre lab. For example, to comment something out I would do:
%  \ifx
%	\begin{itemize}
%		\item item1
%		\item item2
%	\end{itemize}	
%  \fi

\addcontentsline{toc}{section}{References}
\bibliographystyle{IEEEtran}
\bibliography{sow}
\end{document}
















